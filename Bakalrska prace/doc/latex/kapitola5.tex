\chapter{Výzkumný záměr}
Javascriptové aplikační rámce jsou v dnešní době při tvorbě webových aplikací čím dál častěji využívány. V rámci JavaScriptu jsou aplikační rámce nejvlivnější technologií a poptávka po vývojářích, kteří s nimi umí pracovat, je vysoká. Aplikace mají díky nim stabilnější strukturu, jsou bezpečnější, modulárnější a dají se snadno škálovat. Díky své obsáhlosti umožňují aplikační rámce velmi rychlý vývoj aplikací. Dříve představené javascriptové aplikační rámce naplňují podobné úkoly, ale v mnoha vlastnostech se od sebe liší, proto je podstatné je srovnat a vymezit jejich použití. Každý rámec se hodí pro jiné účely, některé jsou například vhodnější pro tvorbu menších aplikací a jiné jsou ideální pro tvorbu aplikací zaměřených na grafiku anebo pro aplikace s rozsáhlou systémovou logikou. V této práci je mým záměrem srovnat dva ze dvou popsaných rámců, konkrétně se jedná Angular a Vue.

Jedním ze způsobů, jak je možné oba aplikační rámce srovnat, je vytvoření aplikace, v níž by bylo možné vidět potenciál popisovaných technologií a rozdíly mezi nimi. Taková aplikace by měla být natolik komplexní, aby její vnitřní logika byla rozsáhlá a aby mohla mít i reálné využití. Jednoduchá testovací aplikace by neumožnila takovou míru srovnání v kritériích, na které se chci zaměřit. Výzkumy využívající totožné testovací aplikace v různých aplikačních rámcích a porovnávající charakteristiky jako je například rychlost zpracování, velikost výsledného souboru ap. navíc již existují. Cílem této práce je však srovnat špatně kvantifikovatelné rozdíly, které jsou spojeny s praktickým a cíleným využitím aplikačních rámců. Na základě poznatků vyplývajících z literární rešerše jsem se rozhodl využít jako hlavní aplikační rámec Angular doplněný o komponenty vytvořené pomocí rámce Vue. Angular je určený pro mohutnější aplikace, má stabilnější architekturu a má v sobě implementovanou jednoduchou komunikaci se serverem pro případnou práci s databází. Vue je narozdíl od toho odlehčený aplikační rámec, který je lehký a rychlý, a bývá běžně využíván k tvorbě univerzálně využitelných komponent.

Pro účely takového srovnání jsem se rozhodl navrhnout a vytvořit webovou aplikaci umožňující správu a prodej lístků na události. Tato aplikace bude rozdělena do tří částí, kterými jsou prodej vstupenek, správa vstupenek a kontrola vstupenek. Prodej vstupenek bude zobrazovat aktuální počet volných vstupenek preferovaně v reálném čase, jeho součástí bude platební brána a funkce umožňující poslat vstupenku uživateli anebo mu ji zobrazit. Správa vstupenek pak bude obsahovat minimálně seznam událostí, na které je možné zakoupit vstupenky, a formulář pro tvorbu nových událostí nebo jejich úpravu či smazání. Část sloužící ke kontrole vstupenek bude přijímat kódy vstupenek. K tomuto bude využita technologie načítání QR kódů a následné zobrazení, zda je vstupenka platná či ne.
