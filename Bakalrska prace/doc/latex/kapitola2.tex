\chapter{Webové aplikace}
Základní definice webové aplikace je taková, že se jedná o program, k němuž se lze připojit pomocí HTTP (Hyper Text Transfer Protocol), a to pomocí počítačové sítě (Internet) ale i v rámci lokální podoby (Intranet). Vymezení toho, co je a co není webová aplikace, je v dnešní době velmi složité, jelikož existuje velké množství definic, které nemusí souhlasit v každém bodě. Webová aplikace může být totiž více či méně rozvinutá, což komplikuje vytvoření jednotné pojetí. Všechny definice se ale shodují v tom, že webová aplikace je rozšíření webové stránky a že se snaží uživateli zlepšit jeho uživatelskou zkušenost a práci s webovou stránkou. Činí tak za pomoci vnitřní logiky, a sice v rozsahu od jednoduchých filtrů nebo ukládání dat o uživateli, která jsou později zpracovávána, až pro vytváření informačních systémů a online obchodů. Dá se tedy říci, že jde především o zefektivnění a zpřístupnění dat uživateli na základě vstupních informací. Rozdíly mezi webovými aplikacemi a pouhými stránkami mohou být především pro uživatele méně rozpoznatelné, jelikož i jednoduché informativní stránky se dnes běžně snaží o efektivnější a příjemnější komunikaci s uživatelem \cite{differencewebapps}.

Jelikož jsou webové aplikace tak rozsáhlým tématem, je vhodné si kromě definice také ujasnit základní pojmy, s nimiž se často spojují, a vymezit si rozdíly a spojitosti mezi nimi. Webová stránka (website) je kolekce webových stránek a souborů, které jsou uloženy na webovém serveru a jsou dostupné pomocí internetu. Tyto kolekce se nachází pod jednou společnou internetovou doménou pojmenovávající tuto kolekci \cite{website}.

Webový server je pak počítač, který se zaměřuje na odpovídání na HTTP požadavky od klienta a jejich zpracovávání do dokumentu, který je reprezentován webovým stránkami. Tyto úpravy provádí s pomocí aplikačního serveru za účelem vytvoření dynamických stránek. Samotný aplikační server je software, jehož účelem je zpracování stránky za pomocí serverových jazyků, a to před tím než bude zaslána klientovi \cite{webserver}. Aplikační server tak zajišťuje dynamický charakter stránky, přístup k databázi a také vnitřní logiku. Nejpoužívanějším jazykem pro tento způsob zpracování je PHP (Hypertext Preprocessor).

Existuje několik druhů stránek, přičemž nejzákladnější rozdělení v kontextu této práce je členění na statické a dynamické webové stránky. Statické stránky mají jasně daný obsah, který server nikdy nemění, a pouze jej při zažádání zašle uživateli. Je ovšem nutné podotknout, že statické stránky se v dnešní době chovají, jako by byly dynamické, a to díky implementaci skriptů, které umožňují pohyblivost stránky, což v jistém ohledu připomíná dynamické stránky \cite{futureofsoftware}. 
Dynamické stránky jsou oproti tomu proměnlivé. Při zažádání o webovou stránku mohou být zaslány v požadavku dodatečné informace, které přimějí webový server, aby poslal upravenou verzi dané stránky. Aplikační server provede úpravy na stránce a zašle uživateli novou podobu stránky. Tyto úpravy často souvisí se zobrazováním dat z databáze, k níž přistupuje webový server. Získává z ní podstatná data, která poté zpracuje a přeposílá zpátky uživateli. Tato data jsou ukládána především ve formě tabulek \cite{futureofsoftware}.

Samotná webová stránka je reprezentována v podobě dokumentů, k nimž se dá přistoupit skrze internetovou síť (tj. zpravidla prostřednictvím WWW). Jako základ se používá značkovací jazyk HTML a stylování pomocí CSS. Stránka se běžně skládá z informačních textů, videí, obrázků, odkazů a dalších prvků.

    \section{Historie tvorby webových aplikací}
Webové aplikace vzešly z přenosu dat pomocí HTTP (Hypertext Transfer Protocol) mezi klientem a serverem. Tento přenos dat uskutečnil Tim Berners-Lee v roce 1989. O rok později byla vytvořena první webová stránka, která popisovala projekt jako takový a obsahovala návod jak takový přenos dat vytvořit. Tím začala éra World Wide Webu (WWW), tedy celosvětové sítě umožňující okamžitou online komunikaci. Jako první se rozšířila tvorba webových stránek, které obsahovaly pouze text a jejichž účelem bylo předávání odborných informací. Mezi prvními informacemi předávanými tímto způsobem byly výsledky vědeckých výzkumů anebo vládní či vojenské dokumentace. Jediným nástrojem, který v té době tvořil tyto stránky, byl značkovací jazyk HTML (Hypertext Markup Language), který se stal základním kamenem pro veškerou tvorbu na webu \cite{futureofsoftware}. Díky inovativnímu prohlížeči Mosaic bylo možné kromě textu a základních tabulek zobrazovat také obrázky, což umožnilo získat zájem většího okolí a naznačilo cestu pro designový vývoj webu. Z odlišných fontů a barevných schémat začal vznikat nový vzhled webu, u kterého začala nabývat na významu také estetická rovina \cite{webbrowsers}. V první půlce 90. let začalo docházet k vyššímu využití potenciálu, který World Wide Web nabízel, a kromě informačních webů došlo i ke vzniku a rozšíření webových stránek vytvořených ke komerčním účelům. Počet uživatelů rapidně vzrostl na desítky milionů. Díky rostoucímu množství uživatelů bylo čím dál dostupnější internetové připojení a žádná společnost nechtěla být pozadu s dobou. V následujících letech umožnily nové technologie rozvoj tohoto odvětví. Došlo ke vzniku nových významných programovacích jazyků, které se staly základem moderního vývoje webových stránek. Mezi tyto jazyky patří CSS, PHP, JavaScript, Flash a mnohé další.

Během roku 2000 se začala hroutit internetová ekonomika, což bylo zapříčiněno prasknutím ekonomické bubliny a následným poklesem hodnoty akcií řady společností, které se zabývaly technologiemi spojenými s internetem \cite{bubble}. Jakmile došlo ke zotavení trhu po této krizi, začalo docházet k vytváření základů, které později umožnily globální rozšíření webu i pro širokou veřejnost. Počítače začaly být čím dál výkonnější, bylo umožněno vytvářet komplexnější webové stránky a využívat multimédia. Tento technologický rozvoj umožnil vyšší efektivitu stránek, častější využívání vizuálních prvků jako jsou animace anebo videa. Jazyky se tomuto tempu vývoje průběžně přizpůsobovaly a jejich vývojáři se snažili, aby programovací jazyky byly efektivněji využitelné. V této době vznikl především jQuery a AJAX, které nachází své využití i v dnešní době. Vznikly také nejpopulárnější stránky hostující webové aplikace, kterými je mimo jiné Youtube, Twitter anebo Facebook. Roku 2007 byla představena možnost používání internetu na mobilních zařízení díky představení prvního zařízení iPhone vyvinutého společností Apple. Tento bod značí začátek nutnosti řešení dynamického rozlišení a vzhledu stránek, vícedotykového ovládání a vytváření mobilní webové aplikace \cite{futureofsoftware}.

Počítače v té době měly výkon, který postačoval k nutným úkonům, ale zaostávaly zprostředkovací webové prohlížeče.Tvorba webových aplikací je s webovými prohlížeči silně spojena proto, že bez prohlížeče není prakticky možné se k aplikacím připojovat. Vývoj nových prohlížečů nicméně nebyl jednoduchý. Tvůrci prohlížečů se snažili nabídnout svá originální řešení a funkce, což znemožňovalo vznik norem, které by zajistily kompatibilitu napříč prohlížeči. Toto komplikovalo vývoj webových stránek a aplikací v té době, a to především proto, že každý z dostupných prohlížečů zobrazoval webovou stránku odlišně. Některé prohlížeče byly například založeny na různých interpretacích standardů a jiné si vytvářely vlastní značení pro HTML jazyk, který ale nebyl kompatibilní s konkurenčními prohlížeči. Každý také zobrazoval i styloval pomocí CSS jiným způsobem, některé příkazy byly chybně implementované, což vedlo ke značným komplikacím v rámci web designu.

Mezi nejpopulárnější prohlížeče v dnešní době patří s velkým předstihem Chrome od společnosti Google \cite{chrome}. Dalšími široce rozšířenými prohlížeči je Safari vytvořený společností Apple, dále Internet Explorer a Edge spadající pod Microsoft anebo Firefox a Opera vytvořené stejnojmennými společnostmi. Internet Explorer je nejstarším ze stávajících běžných prohlížečů. V druhé polovině 90. let soupeřil s prohlížečem Netscape Navigator, Internet Explorer jej ale předčil technickými vymoženostmi, například první implementací CSS neboli kaskádových stylů. Roku 2002 se objevil nový konkurenční prohlížeč Firefox. Ten byl sice hodnocen kladně, ale Internet Explorer byl tehdy rozšířenější díky popularitě operačního systému Windows, který spadá pod stejnou společnost. V roce 2008 ovšem Google představil svůj prohlížeč Google Chrome, který překonal v následujících čtyřech letech veškeré konkurenční prohlížeče. Chrome dokázal efektivně zobrazovat webové stránky a aplikace a především zvládal rychle načítat JavaScript a dosahoval vysokého výkonu při načítání stránek, což mu dodalo potřebnou konkurenční výhodu. Tento prohlížeč zvládal standardizované testování ekma skriptem a W3C standardem, a to především díky jádru V8, které pohání javascriptovou část prohlížeče Chrome. Na základě vytvoření a rozšíření prohlížeče podporujího JavaScript v jeho plném rozsahu následně došlo ke zlepšení využitelnosti JavaScriptu a vytvoření podmínek, které umožnily tvorbu interaktivních a plynulých stránek tak, jak je dnes známe \cite{futureofsoftware}.

Od roku 2008 až po současnost tak došlo k rychlému rozvoji technologií a web developmentu, díky čemuž jsou stránky responzivní, fungují také offline, mají cloud úložiště, objevuje se HTML5 s novými prvky, CSS 3 přináší animace, přechody anebo roztahovatelné panelování. Dále se objevila knihovna Bootstrap umožňující snadnější stylování a k vývoji webových aplikací jsou čím dál častěji využívány jazyky z oblasti serverů jako je například Python, Ruby nebo JavaScript. Velmi oblíbenými se stávají rovněž společnosti, které nabízí předpřipravené šablony webových stránek a aplikací - mezi takové patří mimo jiné Wordpress. V dnešní době tak existuje řada nástrojů, které umožňují rychlejší a kvalitnější vývoj webových stránek i aplikací. Webové aplikace v posledních letech zaznamenaly právě i díky jejich dostupnosti velký nárůst v popularitě. Postupně nahrazují desktopové aplikace a jsou již běžnou a podstatnou součástí internetu. Stávají se také klíčovou součástí v podnicích, kde jsou využívány například jako informační systémy nebo internetové obchody \cite{understandwebapps}. Webové aplikace nyní nachází své uplatnění mnohdy také jako uživatelská rozhraní pro konfiguraci a správu velkého množství zařízení jako jsou směrovače, čidla, chytrá zařízení, chytré domovy, servery, kamerové systémy anebo domácí spotřebiče.

    \section{Vlastnosti webových aplikací}
Webové aplikace se stávají populárnější především díky jejich univerzálnosti. Jejich použití vyžaduje na straně klienta pouze některý z prohlížečů, které mají standardizované jazyky na zpracování kódu, díky čemuž jsou webové aplikace funkční skrze různé platformy přístupu (Windows, IOS, Android, Linux). Tato vlastnost velmi výrazně usnadňuje práci a dovoluje tak programátorům, aby se zaměřovali na specifičtější problémy a nemuseli být specializovaní v mnoha různých oborech a jazycích jen proto, aby byli schopni zajistit, že aplikace bude dostupná pro všechny platformy. Toto se stává podstatným právě díky zvyšující se oblíbenosti iOS produktů a zpřístupnění Linux operačních systémů jako je například Ubuntu Debian obyčejným uživatelům. Díky tomu vznikl nový typ programátorů, který se místo zaměření na různé platformy zaměřuje na specifické vrstvy webových aplikací a získává tak větší míru specializace v dané oblasti. Rozšíření webových aplikací také vedlo k dalšímu rozvoji programovacích jazyků využívaných pro webové aplikace. Například JavaScript byl vytvořen především jako jazyk pro vzhledovou úpravu stránky a interpretaci HTML kódu, ale stal se i jazykem pro serverovou vrstvu aplikace v podobě prostředí Node.js. Node.js dokáže vytvářet interní logiku a komplexní operace, k čemuž JavaScript původně ani nebyl zamýšlen.

Webové aplikace provádí svou vnitřní logiku především na serveru. Existují ale i aplikace, u kterých dochází ke stáhnutí části programu do počítače, a část úkonů či ukládání souborů je provedeno na straně klienta. Tento proces lze také rozšířit a vytvořit na jeho základě webové aplikace, které je možné nainstalovat přímo do zařízení, a to především v rámci systému Android ve formě Progressive Web Apps. Tyto aplikace částečně anebo úplně fungují bez internetového připojení.

Další z velmi důležitých vlastností webových aplikací je jejich jednoduchá přístupnost, jejich zálohovatelnost a možnost mít je uložené na více serverech. Toto pak redukuje možnost, že bude aplikace nedostupná. Samotné ukládání dat u webových aplikací probíhá na jednom místě. Není tak nutné se zpravidla dotazovat na více lokacích, jako je tomu například u desktopových aplikací, kde může být část dat uložena lokálně a část může být získatelná přes internet, čímž dochází ke zbytečné komunikaci s větším množstvím úložišť. U webových aplikací jsou data často ukládána na tzv. cloudu, což je výkonný server, který dokáže zpracovávat dotazy na data velmi rychle a dokáže je správně distribuovat mezi klienty. Tato serverová úložiště jsou také více spolehlivá, jelikož jsou na strategických místech s kvalitním zabezpečením. Většina úložišť je zde navíc věnována přímo datům, takže je jejich zpracovávání serverem mnohem efektivnější. Dokáží také provádět zálohování dat na jiné servery během toho, co jsou všechny klientské počítače vypnuté. Během noci tak dojde například k zálohování a následně jsou data dostupná na více místech a ve více verzích \cite{modernapps}.
Díky tomu, že webové aplikace klient nemusí instalovat na své zařízení, je jejich údržba i vývoj velmi jednoduchý. Je-li zjištěn jakýkoli problém s aplikací, může vývojář přímo pracovat na vývojové verzi aplikace a následně ji bez prodlení a výrazných komplikací na straně uživatele aktualizovat. Tato vlastnost také umožňuje jednoduché vytváření verzí systému, kdy může vývojář živě ukazovat uživateli stránky možnosti vzhledu aplikace a dle potřeb klienta je v rámci vteřin upravovat. Na tyto aplikace se mohou připojit tisíce uživatelů současně a bez komplikací vytvářet ve stejnou dobu stovky dalších požadavků. Tento princip lze využít také k tzv. real-time úpravám, tedy k současnému upravování obsahu na stránce z více klientských zařízení. Podobně funguje například Google Sheets. Tento systém úprav umožňuje koordinovanější práci a rychlejší reakci na změny oproti aplikacím, které real-time úpravy neumožňují.

Další charakteristikou webových aplikací je to, že je jejich vzhledová část jednoduše programovatelná. Umožňují to snadné značkovací jazyky a rozmanité možnosti stylování, kterou jsou současně vysoce intuitivní. Toto současně s tím, že logika webové aplikace může být naprogramovatelná v téměř jakémkoli jazyce, je jedním z důvodů, proč webové aplikace bývají mnohdy levnější a jednodušší na vytvoření \cite{characteristics}.

Mnoho těchto logických operací je při tvorbě webových aplikací vytvářeno na serverech, které jsou specializovány na dané úkoly. Není tak nutné při zálohování, verzování dat anebo komplexnějších výpočtech zatěžovat zařízení klienta, které tak pouze přistupuje k online datům.

Přestože webové aplikace disponují řadou přínosných vlastností, nelze vynechat ani negativní aspekty jejich využití. Jednou z takových nevýhod je fakt, že webové aplikace kvůli své univerzálnosti zpravidla nemohou bez podpory třetích stran přistupovat k výpočetní síle klienta ani k velkému množství hardwaru. To znamená, že programy vyžadující náročné výpočty jako je například zpracovávání videí, grafiky, spouštění her, se nemůžou dostat k potřebným prostředkům pro hladký provoz. Další nevýhodou je, že pro využití těchto aplikací je potřeba připojení k internetu, které se ale stává méně problematické díky ukládání dat na straně klienta a kvalitním internetovým pokrytím. Je zde také problém s častými útoky na webové aplikace za účelem získání osobních dat anebo znepřístupnění či zneužití aplikace \cite{modernapps}.

    \section{Typy webových aplikací}
Za webovou aplikaci lze označit širokou škálu softwarů, které běží na internetovém serveru a k nimž se připojuje pomocí prohlížeče. Webové aplikace lze nicméně rozdělit na tři základní typy s podobnými charakteristikami. Jedná se o tradiční webové aplikace, single-page aplikace a progresivní webové aplikace, které se od sebe liší tím, nakolik připomínají nativní aplikace.

        \subsection{Tradiční webová aplikace}
V kontextu této práce lze za tradiční webové aplikace považovat převážně aplikace, kde se každá stránka načítá zvlášť. Jde tedy o aplikace statické, které byly dynamicky vytvořené na serveru, nebo dynamické aplikace s menším množstvím dynamických prvků. Tyto tradiční aplikace se převážně skládaly a skládají z HTML, CSS a občas také z JavaScriptu. Protože jsou takové stránky převážně statické, nezatěžují klienta se zpracováním dat, veškeré procesy totiž řeší server. Tyto aplikace se rychle načítají díky jejich malé velikosti a jednoduchosti. Problémem je ale již zmíněné načítání stránek při navigaci v samotné aplikaci, kdy se každá stránka musí načíst samostatně. Ve spojení s malým množstvím dynamických prvků to způsobuje, že uživatelské rozhraní aplikace nepůsobí jako celek. Při častějším navigováním mezi stránkami aplikace uživatel opakovaně čeká, což vede ke zhoršení uživatelské zkušenosti s takovou aplikací. V dnešní době se od těchto praktik upouští skrze dobrou úroveň výpočetní techniky na straně klienta a zefektivnění webových jazyků jako je především JavaScript. Celkově je kladen větší důraz na snahu zlepšit požitek uživatele z interakce s webovou aplikací \cite{spadevelopment}.

        \subsection{Single-page aplikace(SPA)}
Jak již název napovídá, single-page aplikace (SPA) neboli jednostránkové aplikace mají jen jednu stránku. Klient požádá server o stránku pouze jednou a po přijetí už nedochází k žádnému dalšímu načítání stránek. SPA obsahují veškeré informace pro to, aby mohl server aktuální stránku přeskládat tak, aby reprezentovala potřebný náhled. Jednostránkové aplikace obsahují základní HTML strukturu a veškerá aplikační logika je obsažena v JavaScript kódu. Tento kód obsahuje informace pro vygenerování požadované stránky přímo v prohlížeči, veškerá práce je tak prováděna na straně klienta. Proces navigování na stránce je řešen pomocí přepsání náhledu stránky. Struktura DOM je přetvařování podle požadavků a potřebná data jsou zpracována na pozadí. Tento přístup zajišťuje rychlejší a plynulejší interakci uživatele se stránkou a tyto aplikace mnohdy obsahují příjemné transformace a animace se kterými může uživatel interagovat. Komunikace se serverem je nadále pouze v pozadí v rámci AJAX požadavků, které většinou získají data ve formátu JSON nebo XML. Tyto data se následně pouze doplní do náhledu podle potřeby a opět veškeré přetváření dat probíhá především na straně klienta. Tento přístup tvorby webových aplikací sebou ale nese i své nevýhody. Při vývoji musíme předpokládat že každý s klientů má dostatečné výpočetní prostředky pro zpracování naší aplikace. Tento problém se vyskytuje čím dál méně u počítačů, ale stále přetrvává u slabších mobilních zařízení, kdy úvodní načítání může trvat delší dobu. Hlavním pohonem těchto aplikací jsou javascriptové frameworky, tudíž je spolu se stránkou zaslána i většina frameworku a spoustu knihoven pro správu aplikace. Existují ale způsoby jak zefektivnit načítání aplikace, jako například Lazy loading, což je načítání modulů s prioritami. Nejprve bude načten modul s frameworkem a hlavní logika pro aplikaci a úvodní stránka, moduly které nejsou okamžitě potřeba jsou načteny až podle potřeby. Další z metod je Tree shaking, tento přístup zajišťuje že většina kódu který není použitý, například neimplementované metody, nepoužité knihovny a podobné, jsou vyřazeny z finální sestavy kódu. Zdrojový kód se tedy může značně zkrátit bez jakéhokoliv negativního vlivu na aplikaci \cite{spadevelopment}. 

        \subsection{Progresivní webové aplikace (PWA)}
Jedná se o webové aplikace, které nabízí funkce běžné pro nativní neboli instalované aplikace. Podstatnými vlastnostmi je možnost používat aplikaci offline, ovšem mnohdy s omezenými možnostmi, a umožnění přístupu k hardwaru zařízení. Tyto aplikace jsou téměř výhradně mobilní. PWA byli definované v 2015 s pravidly, koncepty a doporučeními kterými by se měla PWA řídit. PWA aplikace by měla být, rychlá, funkční v jakémkoliv prohlížeči, obsahovat offline verzi aplikace, responzivní pro jakékoliv rozlišení, komunikace je výhradně pomocí HTTPS, musí být instalovatelná na zařízení, měla by obsahovat Manifest soubor se základními informacemi o aplikaci, měli by se chovat jako nativní aplikace, a také by měli dodržovat obecně dobré praktiky pro tvorbu webových aplikací. Pro zajištění těchto vlastností bylo zapotřebí nových technologií. Jednou z těchto technologií je Service worker, který umožňuje zpracování oznámení uživateli, synchronizují data na pozadí. Dokáží se přizpůsobit jestli je uživatel připojen k internetu nebo jen s omezeným přístupem internetu a nebo že je offline. Efektivně pracují s ukládáním dat v mezipaměti pro možný výpadek internetu. Další podstatná technologie umožňují existenci PWA je IndexedDB. Jedná se o prohlížečové rozhraní umožňují správu NoSQL databáze, díky které je možné lépe designovat datovou strukturu uvnitř databáze. Tato databáze a technologie IndexedDB společně umožňují ukládání databází lokálně v zařízení a práci s nimi jako by byly na serveru \cite{pwas}.

    \section{Základní technologie}
        \subsection{HTML a HTML5}
Prvními jazyky pro tvorbu webu byly od počátku HTML a CSS, které za dobu své existence prošly velmi dynamickým vývojem. Hypertext Markup Language (HTML) je značkovací jazyk, který určuje obsah stránky jako je například text, odkaz nebo obrázek. HTML v dnešní době převážně neurčuje vzhled stránky, ale definuje prvky, které jsou nadále upravovány stylovacím jazykem CSS. Některé značky obsahují vlastnosti, které blíže specifikují význam značky, jako je třeba název nebo typ obsahu. Nástupcem HTML 4 se měl převratem tisíciletí stát značkovací jazyk XHTML, který obsahuje kombinaci jazyků HTML a XML. Tento jazyk se nepodařilo rozšířit a namísto něj odpůrci vytvořili návrh jazyka HTML 5, který přinesl mnohem více, než se doufalo od XHTML. Tato nová verze se stala světově uznávanou, opravila mnoho chyb předchozí verze a podporuje tvorbu webových aplikací novými technologiemi. HTML 5 nově vnitřně podporuje video a audio formát, vektorovou grafiku a především výrazně podporuje práci s JavaScriptem natolik, že přimělo vývojáře prohlížečů, aby lépe zpracovávali JavaScript ve svých prohlížečích \cite{spadevelopment}. V kombinaci s JavaScriptem může prohlížeč využívat periferie počítače, což předtím nebylo možné. Toto usnadňuje práci s lokací, kamerou, mikrofonem a dalšími periferiemi, a to bez použití rozšíření třetích stran (např. Flash Player či Java), jako tomu bylo dříve. HTML 5 a JavaScript také přináší nové značky, které lépe definují rozdělení stránky. Příkladem takových značek je například “nav” neboli navigace, “header”, což je záhlaví stránky s názvem a logem, dále “footer”, tzn. zápatí stránky, “article” značící samostatný blok, který nese sám o sobě smysl, anebo “section”, což je obecná sekce. Tyto a další značky nahradily nesčetné množství tzv. “div” elementů s různými třídami. Přibylo rovněž lokální úložiště s větší kapacitou a možnost práce s databází.

        \subsection{CSS}
Pro vzhledové úpravy HTML je zapotřebí použít Cascading Style Sheets (CSS), jak již bylo nastíněno. Jazyk HTML byl původně navržen tak, aby nebylo zapotřebí dalšího jazyku pro vzhledové úpravy. Způsob implementace ale nebyl optimální a použití jiného jazyka umožnilo oddělení vzhledu dokumentu od jeho obsahu. CSS dokáže upravovat elementy podle typu, podle přiřazené třídy nebo atributu, podle identifikátoru anebo dle jejich různých kombinací a vnoření. Následně lze specifikovat i dynamičtější vlastnosti jako je pozice ve stromu elementů, označení, zda byl prvek již použit nebo zda má uživatel kurzor na objektu \cite{eloquentjava}.

Co se týká struktury, jazyk CSS je kaskádový, což znamená, že pravidla se v něm kombinují a změna stejné vlastnosti se přepíše na základě priorit. V rámci CSS je dále velmi podstatný koncept dědění vlastností z rodičovských elementů. Při použití CSS nelze mluvit o malých vzhledových úpravách jako je změna barvy nadpisů nebo přidání odsazení textu. Při využití svého plného potenciálu je CSS naopak robustní a silný designový nástroj \cite{learningdesign}.

        \subsection{DOM}
Pro umožnění komunikace mezi JavaScriptem a HTML dokumentem je zapotřebí DOM. DOM neboli Document Object Model je reprezentace HTML nebo XML dokumentu do stromové struktury s určitou hierarchií. Prohlížeč sestaví tento strom ihned poté, co získá dokument od serveru a uloží si jej do paměti. Díky této reprezentaci je možné k dokumentu přistupovat dalšími jazyky a měnit jeho strukturu, vzhled a obsah. Strom je složen z uzlů, které přechází do větví a každá větev stromu končí listovým uzlem, který obsahuje objekty. Uzly mohou mít k sobě navázané obsluhy událostí, což znamená, že po kliknutí na prvek se spustí předem definované funkce \cite{domintro}. Struktura stromu má jeden základní kořen, nemá žádné cykly a v rámci uzlů nemohou existoval další uzly. Uzly obsahují číslo označující jejich typ a ukazatele na rodičovský uzel a uzly potomků a sourozenců. DOM byl původně navržen pouze pro spolupráci s JavaScriptem a jejich vývoj byl na sobě závislý. Jelikož však bylo zapotřebí určit standardy pro web, byly obě technologie odděleny a DOM byl zobecněn \cite{learningdesign}.
        \subsection{Nástroje pro zvýšení efektivity programování}
Rád bych zmínil několik technologií, které významně pomáhají tvorbě webových aplikací, přičemž ne všechny jmenované jsou využitelné pouze pro web. Mezi tyto nástroje patří například knihovny (například Bootstrap), aplikace pro psaní kódu (IDE) a verzovací programy, o kterých budu psát níže. Nástrojů pro zvýšení efektivity programování je samozřejmě mnohem více, ale pro účely této práce bude postačovat základní výčet.

Prvním nástrojem zvyšující efektivitu jsou knihovny, které jsou tvořeny různými funkcemi, metodami nebo objekty k provádění opakovaných obecných úkolů na webové stránce nebo aplikaci. Knihovny mají kód nebo funkce, které mohou vývojáři znovu použít a nebo je využít jako základ pro jejich funkce. To zajišťuje, aby vývojář nemusel nutně psát kód k určitým úkolům pokaždé od začátku, ale aby namísto toho mohl použít knihovnu, která implementuje řešení pro jeho problém v podobě komplexnějších funkcích. Existence knihoven podněcuje programátory, aby sdíleli svá řešení s komunitou a tím se vytváří prostředí, díky kterému není nutné ztrácet čas na základních problémech. Vývojáři se tak mohou soustředit na komplexnější problémy, zvyšuje se efektivita práce a pokrok v rámci oboru.Tento princip je pak dále rozšířen do aplikačních rámců, které dále lépe strukturují vývoj aplikací a usnadňují jejich vývoj.

Jednou z nejpopulárnějších knihoven pro tvorbu na webu je Bootstrap. Jedná se o knihovnu obsahující komponenty v HTML, CSS a JavaScriptu. Jejím zaměřením je zjednodušení vývoje webových stránek a aplikací. Bootstrap poskytuje základní definice stylů pro všechny prvky HTML, které dodávají aplikaci ucelený vzhled. Obsahuje také velké množství předdefinovaných tříd s responzivními vlastnostmi a s hodnotami proporciálně vhodnými pro úhlednou webovou aplikaci. Tím je míněn například vzhled tlačítek, velikost textu, odsazení mezi paragrafy a mnohé další. Bootstrap přináší také zjednodušený systém rozložení stránky, definuje nové elementy jako jsou informační nápovědy a dialogové okna a rozšiřuje funkce některých elementů (např. automatické doplňování ve formulářích). Jedná se o nástroj, který výrazně zkracuje dobu strávenou na programování designu a rozložení stránky či aplikace \cite{bootstrapbook}.

Dalším nástrojem přispívající k vyšší efektivitě programování webových aplikací je integrované vývojové prostředí (IDE). Je to software pro vytváření aplikací, který kombinuje běžné vývojářské nástroje do jediného grafického uživatelského rozhraní (GUI). IDE obvykle tvoří editor kódu, kompilátor, debugger, prohlížeč tříd a další pomocné nástroje. Jak napovídá název, editor kódu je určený pro psaní a úpravy zdrojového kódu. Editory zdrojového kódu se odlišují od textových editorů tím, že jsou záměrně vytvořené ke zjednodušení a zlepšování psaní a úpravy kódu. Kompilátory jsou pak nástrojem, který transformuje zdrojový kód napsaný v lidském čitelném či zapisovatelném jazyce do podoby spustitelné počítačem. Debugger se pak používá během testování, a to konkrétně k ladění aplikačních programů. Pod IDE spadají také automatizační nástroje automatizující běžné úlohy vývojářů. Efektivitu zvyšují také prohlížeče tříd, které se používají ke zkoumání a odkazování na vlastnosti objektově orientované hierarchie tříd.

Posledním nástrojem, který zde zmíním, jsou verzovací systémy. Jedná se o systém pro systematické ukládání změn provedených v aplikaci, který umožňuje psát komentáře ke změnám, vracet se k předchozím verzím kódu, rozdělit části vývoje aplikace do větví, které jsou po dokončení vývoje spojeny. Tento nástroj tedy velmi ulehčuje práci v týmu a umožňuje se vrátit do minulých stavů aplikace.
