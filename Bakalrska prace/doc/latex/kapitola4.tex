\chapter{Javascriptové aplikační rámce}
Nejpodstatnější technologií spojenou s JavaScriptem jsou ale aplikační rámce neboli frameworky. Tyto aplikační rámce v sobě spojují dříve popisované technologie a navíc přidávají mnoho dalších funkcí, které usnadňují tvorbu aplikací. Javascriptové aplikační rámce jsou nezbytnou součástí moderního webového vývoje front-end, neboli prezenční část aplikace běžící na straně klienta, a poskytují vývojářům osvědčené nástroje pro vytváření škálovatelných interaktivních webových aplikací. Rámce jsou typem nástroje, díky kterému je práce s JavaScriptem jednodušší a plynulejší. V jejich nejzákladnějším prostředí jsou rámce kolekcí knihoven kódů JavaScriptu, které vývojářům poskytují předem napsaný kód, který lze použít pro rutinní programovací funkce a úkoly - tedy doslovně rámec pro vytváření webů nebo webových aplikací. 

Výhoda v používání aplikačních rámců JavaScriptu je celková efektivita a organizovanost, které do projektu přinášejí. Frameworky často obsahují efektivnější a bezpečnější implementace některých funkcí jako je například bezpečné ukládání dat a efektivní užití knihoven. Kromě toho rámce navádějí na přehledné strukturování kódu a poskytují hotová řešení pro běžné problémy s kódováním. Na druhou stranu může být celá tato struktura nevýhodou v tom, že vyžaduje dodržování pravidel a konvencí specifických pro daný framework, což omezuje svobodu, kterou programátorovi nabízí úplně ruční kódování. Částečnou volnost při strukturování projektu nabízí využití knihoven, které rovněž obsahují velké množství předdefinovaných technologií. Rozdíl mezi knihovnou a aplikačním rámcem je tedy to, že aplikační rámec má přesně definovanou strukturu, kterou se vývojář musí řídit, a obsahuje téměř veškeré technologie potřebné pro úplný vývoj aplikace včetně sestavování aplikace, jejího testování, vytváření síťových požadavků apod. Knihovny také nabízí tyto nástroje, ale jejich využití je často podmíněno použitím knihoven a komponent třetích stran. Knihovny nemají striktní pravidla pro strukturu aplikace a použití nabízených technologií je zcela volitelné.

Podrobnější informace ohledně tří nejpopulárnějších javascriptových aplikačních rámců budou představeny v následujících podkapitolách. Jedná se konkrétně o Angular, Vue a React, přičemž rámci React se budu věnovat pouze stručně, jelikož ve své práci se zaměřuji především na srovnání Angularu a Vue \cite{clientsideframe}.

    \section{Angular}
Prvním javascriptovým aplikačním rámcem, kterému zde bude věnována pozornost, je Angular. Jedná se o aplikační rámec vytvořený společností Google v roce 2016. Před touto verzí existoval Angular.js, který byl ve všech aspektech komplikovanější verzí stávajícího Angularu. Angular je plnohodnotný aplikační rámec, což znamená, že obsahuje veškeré technologie potřebné pro sestavení plně funkční aplikace, má jasně definovanou architekturu a obsahuje i nástroje pro testování aplikace. Rámec Angular využívá technologii RxJS umožňující pracovat s datovými toky, které obsahují asynchronní data, tyto data se mohou dynamicky měnit v reálném čase. V Angularu jsou definována jako pozorovatelné proměnné neboli observables a dají se filtrovat, mapovat a zpracovávat. Tato data mohou být závislá na třetí straně, která je ve chvíli změny, předá aplikaci k dalšímu zpracování. Pozorovatelné proměnné tak v danou chvíli nemusí být nutně součástí kódu, ale ve chvíli, kdy začnou existovat, spustí se předem definované procesy \cite{angularforenterprise}.

Angular používá pro předlohu jazyk HTML a pro veškerou vnitřní logiku využívá TypeScript, přičemž TypeScipt je kompilován do čistého JavaScriptu. Šablona je převážně napsána v HTML a přijímá téměř všechny značky HTML až na značku “<script>”. Tato značka je ignorována, aby nedocházelo k vkládání cizího kódu a ke zneužití aplikace. Angular umožňuje zobrazení dat v šabloně pomocí dvojitých složených závorek {{data}}.

Rámec Angular podporuje obousměrnou datovou vazbu, což je mechanismus pro komunikaci šablony s komponenční logikou a naopak. Tato komunikace může být obousměrná, ale i jednosměrná pro obě varianty. 

Angular dále používá inkrementální DOM pro aktualizaci náhledů. Všechny komponenty projdou kompilací, neboli převedením do setů instrukcí. Tyto instrukce jsou poté použity pro vytvoření DOM stromu, sady instrukcí jsou následně porovnány a rozdíly jsou upraveny podle potřeby. Popisovaná DOM struktura díky kompilaci umožňuje proces zpracování zvaný tree-shaking, který u virtuálních DOM struktur není možný. Tree-shaking je proces, kdy se zkontroluje, zda jsou vygenerované instrukce využity a pokud nejsou, jsou během kompilace smazány. Díky tomu, že není nutné generovat virtuální DOM, jako je tomu u ostatních aplikačních rámců, šetří Angular místo v paměti a ve spojení s procesem tree-shaking zajišťuje, že je použito minimální množství paměti. Tato vlastnost je velmi výhodná při použití na mobilních aplikacích Problematické je však to, že se tím prodlužuje kompilační doba náhledu. 

Architektura Angular aplikace se opírá o určité elementární koncepty. Základními stavebními bloky aplikačního rámce Angular jsou komponenty neboli components. Tyto komponenty jsou poté organizovány do modulů NgModules. Moduly shromažďují související kód do sad plnících určitou funkci. Angular aplikace je pak definována skupinou těchto modulů. Aplikace má vždy kořenový modul, který umožňuje sestavení aplikace. Kořenový modul je běžně pojmenován AppModule a poskytuje mechanismus bootstrap, který aplikaci inicializuje a spouští. Tento modul může také obsahovat neomezenou hierarchii podřízených modulů. Moduly jsou často organizovány do logických celků pro větší přehlednost při vývoji větší aplikace \cite{angularforenterprise}.

Komponenty definují a spravují pohledy, což jsou skupiny elementů určené pro zobrazení, ze kterých si Angular může vybírat a upravovat je podle logiky a dat programu. Hlavní struktura pohledu je definována v šablonách. Šablony jsou většinou odděleny od samotné komponentové logiky a jsou umístěny do stejnojmenných souborů s koncovkou “.html”.
Komponenty používají služby (services), které poskytují konkrétní funkce, které přímo nesouvisí s pohledy. Mnohdy jsou to tedy obecné nástroje pro zpracování dat, které mohou být použitelné i v jiných komponentách. Tyto služby mohou být vloženy do libovolného počtu komponent. Kód je díky tomu modulární, opakovaně použitelný a efektivní. Moduly, komponenty a služby jsou třídy, které používají dekorátory. Tyto dekorátory určují jejich typ a poskytují potřebná metadata pro zpracování Angularem.

Každá komponenta má jakýsi životní cyklus, který se skládá z fází, ve kterých se dá s komponentou manipulovat. První ovlivnitelná fáze nastává před samotnou inicializací, používá se na k tomu hák ngOnChanges, který se provede před inicializací a poté rovněž vždy, když dojde ke změně některých dat. Poté přichází inicializace ngOnInit - tento hák se provede při samotné inicializaci. Po každé změně je spuštěn ngDoCheck, který slouží k zachycení změn, které Angular nemusí zaznamenat. Po inicializaci a vedlejší kontrole přichází inicializace obsahu ngAfterContentInit a jeho kontrola prostřednictvím háku ngAfterContentChecked. V tomto bodě dochází ke spuštění inicializace vnořených komponent. Dále po úplné inicializaci náhledů jak kontrolovaných komponent, tak komponent vnořených, je možné se připojit s hákem ngAfterViewInit a jeho kontrolou ngAfterViewChecked. Tímto je inicializace dokončena. Pokud jsou data změněna, přichází znovu inicializace a její kontroly a inicializace vnořených komponent. Jakmile už komponentu není potřeba dále vykreslovat, nastává fáze, během níž je komponenta a všechny vazby s ní spojené (například sledované proměnné, vnořené komponenty apod.) smazána a zde se dá navázat pomocí háku ngOnDestroy \cite{angularforenterprise}.

    \section{Vue.js}
Dalším aplikačním rámce JavaScriptu, který zde popíši, je Vue.js. Jedná se o technologii, která má zachovány podobnosti s rámcem Angular, ale blíží se také knihovně React, je to tedy něco na pomezí obou. Vue.js může být využíván jako odlehčená knihovna, ale může být použit i jako plně funkční aplikační rámec. Vue vytvořil Evan You, který se před prací na Vue podílel na vývoji Angularu. Angular považoval za příliš komplexní, ale viděl potenciál v některých jeho vlastnostech, rozhodl se proto extrahovat části, které na Angularu nejvíce doceňoval, a vytvořil z nich lehkou knihovnu. Ačkoli You zahájil tvorbu Vue při práci s Angularem, považuje Vue za nejpodobnější s knihovnou React, protože jeho základní myšlenkou je datová vazba a komponenty, což je základní funkcí Reactu. Architektura Vue se zaměřuje na vyobrazení náhledů a strukturu tvořenou z komponent s vlastními instancemi Vue. Vue je sám o sobě velmi jednoduše rozšiřitelný, a tak se hodí na tvorbu škálovatelných aplikací. Základní knihovna Vue se zaměřuje pouze na zobrazovací vrstvu, ale ostatní rámcové nástroje jsou v oficiálně udržovaných knihovnách a balíčcích jako je například Vue Router pro přesměrovávání v rámci aplikace nebo balíček Vue Server Render, který umožňuje generování Vue na serveru \cite{vuejsup}. 

Syntaxe Vue.js funguje na principu použití předloh v HTML jazyce, na kterou se navazuje skriptová část v JavaScriptu obsahující metody, data apod. Tyto dvě vrstvy jsou vzájemně propojené a komunikují spolu.. Data definovaná v javascriptové části aplikace lze vázat do HTML pomocí dvojité složené závorky podobně jako v Angular, tedy {{data}}.

Data komponenty jsou uloženy ve specializované funkci “data”, která vrací tato data jako anonymní funkce. Data mohou být mezi komponentami přeposílána za pomoci specializovaného objeku “props”. Každá takto předaná hodnota se stává novou instancí té hodnoty. S takto předanou hodnotou se dá pracovat stejně jako s hodnotou definovanou v “data” funkci.

Architektura aplikace Vue je tvořena komponentami, kdy každá komponenta reprezentuje vlastní instanci Vue. Každá komponenta musí obsahovat předlohu HTML a každá předloha se skládá z pouze jednoho kořenového elementu, uvnitř kterého je obsažen zbytek předlohy. Díky tomu je každá komponenta znovupoužitelná. Každá komponenta a tedy i instance Vue přijímá stejné vlastnosti (například data, zpracované hodnoty, metody a hákování na životní cyklus instance). Jedinou výjimkou je prvotní instance Vue, kterou je takzvaný “root” - tato instance má více možností jako například “el”, která odkazuje na kořenový element v DOM. Veškeré komponenty musí být zaregistrovány buď globálně, nebo lokálně. Vue aplikace se často skládá ze stromu obsahujícího komponenty, přičemž tyto komponenty se mohou znovu používat a zanořovat \cite{vuejsup}.

Vue oproti Angularu využívá virtuální DOM pro aktualizaci stránky. Vue se po sestavení předlohy připojí k DOM a sleduje vlastní data za účelem detekce případné změny. Pokud se hodnoty změní, Vue zpracuje koponentu a promítne ji na virtuální DOM. Poté srovná předchozí virtuální DOM s novým a změnu aplikuje na reálný DOM \cite{vuejsup}. Během tohoto procesu dochází k jistým fázím, které se nazývají životní cyklus, a Vue umožňuje dělat úpravy v daných fázích za pomoci hákování na tyto události v životním cyklu. Před vytvořením nové instance Vue je možné využít hák beforeCreate, který provede kód před vytvořením instance. Vue po dokončení tvorby instance zkontroluje, zda je nastaven příznak “el”. Pokud tomu tak je, pokračuje systém dále ve zpracovávání. Jestliže tomu tak není, bude instance Vue navázaná na rodičovský HTML element. Následně je zpracována HTML předloha. Dalším krokem je připojení instance k DOM, přičemž před provedením připojení je možné zavolat hák beforeMount a po připojení hák Mounted. Dále se pak opakovaně provádí kontrola dat, pokud předtím proběhly nějaké změny, a je možné se navázat na událost pomocí háku beforeUpdate. Dalším krokem je aktualizace virtuálního DOM a je možné se navázat skrze hák Update. Tento proces se opakuje do zavolání destruktora instance. Před samotným smazáním instance je možně použít beforeDestroy a po smazání instance se lze navázat hákem Destroyed \cite{vuelifecycle}.

    \section{React}
React nebo také ReactJS je v podstatě pouze javascriptová knihovna sloužící k vytváření frontendu webové aplikace. Je velmi odlišný od Angularu svou rozsáhlostí a komplexností. React byl vytvořen společností Facebook v roce 2013 jako reakce na rostoucí význam internetových reklam. Facebook chtěl mít k dispozici nástroj k jednoduššímu managementu těchto reklam. Podobně jako v případě jQuery se nejedná o framework, ale o velmi rozsáhlou knihovnu, která obsahuje takové množství technologií, že její používání významně ovlivňuje strukturu aplikace, čímž se podobá aplikačnímu rámci. React se vyznačuje tím, že jsou jeho konečná řešení rychlá, do určité míry jednoduchá a velmi jednoduše rozšiřitelná. Knihovnu React lze také přidat do existujícího projektu napsaného pomocí jiného nástroje. 

React obsahuje velké množství podpůrných knihoven umožňujících snazší vývoj aplikace. Zaměřuje se ovšem pouze na prezenční vrstvu aplikace, tudíž je při tvorbě s využitím Reactu nutné doplnit knihovny od třetích stran (například pro navigování nebo získávání dat). React slouží nejlépe k tomu, k čemu byl navržen, a je proto velmi simplistický a efektivní v modelování a práci s náhledy aplikace. Oproti Vue a Angularu si React zakládá na spojení náhledové logiky a samotného HTML. Pro usnadnění takového splynutí React přináší novou technologii rozšiřující JavaScript, a sice JSX. JSX je zkratka pro JavaScript XML. Jedná se o syntaktické rozšíření JavaScriptu. Konečným produktem JSX jsou prvky, které nejsou ani řetězci, ani součástí kódu HTML, ale kombinací obou. K libovolné proměnné tedy můžeme přiřadit hodnotu, která obsahuje prvky HTML. Pak React zajistí, aby tyto proměnné nebo části kódu byly aplikovány na DOM \cite{reactintro}.

Hlavním stavebním blokem aplikačního rámce React je komponenta. React využívá virtuální DOM, díky čemuž je upravena jen potřebná část stromu DOM. Každá komponenta je buď funkce, nebo třída, která musí obsahovat metodu vykreslení. Komponenty mohou přebírat data z rodičovské komponenty a tato data jsou uložena v objektu “props” a nelze je upravovat. Komponenta dále obsahuje objekt obsahující stav komponenty zvaný “state”. Pokud je tento stav změněn, dojde k opakovanému vykreslení komponenty. Data v Reactu jsou narozdíl od Angularu vázána pouze jedním směrem, tedy že logika kódu ovlivňuje náhled ale ne naopak. Aby proběhla změna opačným směrem, musí nastat událost (například kliknutí na tlačítko). Podobným způsobem jsou zpracována data předaná mezi komponentami - tedy vždy jen z nadřazené komponenty do vnořené komponenty. Pro zpětnou komunikaci je možné využít zpětného volání funkce předaného v parametrech komponenty \cite{reacttutorial}. V rámci Reactu není potřeba mluvit o životním cyklu aplikace, protože není používán. Proces vykreslování komponent je podobný jako ve Vue.
