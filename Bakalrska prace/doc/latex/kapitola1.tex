
\chapter{Úvod}
Webové aplikace prošly v posledním desetiletí dynamickým vývojem a dnes se stávají podstatnou součástí webu a v širším pojetí také společnosti. Stávají se čím dál častějším typem aplikací a nahrazují nativní desktopové aplikace. Současně mají také svůj podíl na zvyšující se kvalitě a funkcionalitě mnohých webových stránek. Je jednodušší naprogramovat stránky tak, aby byly dynamické, aby se přizpůsobovaly uživateli, nezatěžovaly jeho hardware a aby zlepšovaly jeho uživatelskou zkušenost s interakcí s webovou stránkou či aplikací. Toto umožňuje mimo jiné rychlý vývoj technologií pro tvorbu webových aplikací, což je jedno z nejrychleji se rozvíjejících odvětví v rámci programování. Tyto technologie zároveň nekladou na programátora vysoké nároky a umožňují mu se zaměřit se na více specifické funkce aplikací. Toto je do velké míry zásluha jazyka JavaScript, jeho knihoven a technologií na něm založených.

Nejvýznamnější technologií v této oblasti jsou javascriptové aplikační rámce. Jedná se o softwarovou strukturu, která usnadňuje vývoj aplikace, a to prostřednictvím nástrojů a metodiky obsažené v těchto rámcích. Umožňují vývoj aplikací tak, že je vnitřní logika implementovaná pomocí JavaScriptu a běží tak na straně klienta. Mezi nejpopulárnější rámce patří dnes Angular, Vue a React. Každý z těchto aplikačních rámců nicméně funguje jiným způsobem a hodí se tak k odlišným účelům.

Právě na aplikační rámce Angular a Vue a na rozdíly mezi nimi jsem zaměřil svou práci. Aplikační rámce se dynamicky vyvíjí, v nedávné době například Angular zásadně změnil způsob kompilace kódu skrze nástroj pro kompilace s názvem Ivy. V této práci tak představuji současnou podobu a fungování rámců Angular a Vue. Zároveň se zde soustředím na aktuální srovnání obou rámců prostřednictvím vlastní webové aplikace. Mnohá srovnání se zaměřují na charakteristiky aplikačních rámců v rámci jednoduchých shodných testovacích aplikacích. V této práci chci nicméně představit a porovnat práci s Angularem a Vue používanými v rámci jedné komplexní aplikace. Toto umožní porovnat oba rámce novým způsobem a nahlédnout do jejich fungování. Hlavní část výsledné aplikace tak byla vytvořena za pomocí rámce Angular, do něhož byla vložena webová komponenta vytvořená pomocí Vue.

V práci se nejdříve zaměřuji na webové aplikace - představuji jejich historii, vlastnosti, typy webových aplikací a základní technologie využívané při jejich tvorbě. Dále se věnuji samotnému jazyku JavaScript, u nějž uvádím základní informace a píši o jeho vlastnostech a technologiích, které jsou s ním spjaté. V další části se zabývám samotnými javascriptovými aplikačními rámci a představuji současné fungování nejvýznamnějších rámců, tedy Angularu, Vue a Reactu. Následující kapitola představuje výzkumný záměr a návrh aplikace. Poté popisuji implementaci samotné webové aplikace a její části. Další kapitola je zaměřena na srovnání obou aplikačních rámců na základě zvolených kritérií, po němž svou práci zakončuji závěrem. 