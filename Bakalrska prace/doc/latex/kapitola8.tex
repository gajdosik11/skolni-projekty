\chapter{Závěr}
Cílem této práce bylo představit nástroje určené pro vývoj webových aplikací fungující na bázi JavaScriptu, navrhnout vhodnou demonstrační aplikaci za použití alespoň dvou nástrojů a tuto aplikaci realizovat. Dále bylo mým cílem srovnat vybrané nástroje na základě vybraných kritérií spolu se zkušenostmi z tvorby webové aplikace. Ná základě těchto kritérií považuji cíle práce za splněné.

Ve své práci popisuji teoretický základ týkající se základních nástrojů a praktik spojených s vývojem webové aplikace. Je zde také popsán jazyk JavaScript, technologie s ním spojené a samozřejmě také samotné javascriptové aplikační rámce pro vývoj, tedy Angular, Vue a React. Na základě rešerše byla navržena demonstrační aplikace pro použití dvou javascriptových aplikačních rámců, a to Angular a Vue. Tato aplikace byla úspěšně implementována jako webový portál pro prodej a správu vstupenek na koncerty a události. Aplikace využívá řady technologií, které aplikační rámce zpřístupňují. Angular zajišťuje komunikaci s databází v reálném čase, zpracovává většinu vnitřní logiky na straně klienta a dodává aplikaci pevnou logickou strukturu. Vue je použit jako rychlá a snadno modulovatelná webová komponenta zpracovávající data o vstupence do responzivní grafické šablony. Oba nástroje jsou dále na základě pozorovatelných aspektů porovnány.
Webová aplikační část pro prodej vstupenek je dostupná pod následujícím odkazem: \url{https://ticketeer-4d3d3.firebaseapp.com/}. Pro demonstraci případného použití se zde aplikace nachází v testovací šabloně. Aplikační část odpovídající za správu událostí a vstupenek je možné najít pod tímto odkazem: \url{https://ticketeer-4d3d3.firebaseapp.com/admin.html}. 

Práce by bylo možné rozšířit o srovnávací aspekty jako je například velikost souborů či rychlost zpracovávání, to by však vyžadovalo implementaci identické aplikace v obou aplikačních rámcích. Aplikace by dále mohla mít lépe nadesignované webové rozhraní, to však nebylo pro potřeby této práce nutné.
